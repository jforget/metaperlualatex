\documentclass[a4paper]{article}
\usepackage[english,frenchb,german,italian,latin]{babel}
\usepackage[utf8]{luainputenc}
\usepackage{luaotfload}
%\usepackage{graphicx}
%\usepackage{textcomp}
%\usepackage[dvips]{hyperref}
\usepackage{breakurl}
\usepackage{perltex}
\usepackage{luamplib}

% marge haute 25.4 - 12 = 13.4 mm
\topmargin=-12mm
\headheight=0mm
\headsep=4mm
% marge basse 297 - (13.4 + 267 + 4) = 12.6mm
\textheight=267mm

% marge gauche 25,4mm,
% marge droite 210 - (25,4 + 173) = 11,6mm
\oddsidemargin=0mm
\marginparsep=0mm
\textwidth=173mm
\columnsep=8mm

%&Latex

\renewcommand{\rmdefault}[0]{ppl}
\renewcommand{\sfdefault}[0]{phv}
\renewcommand{\ttdefault}[0]{pcr}

\newcommand{\datenum}{\number\year-%
\ifnum\month<10\relax0\fi\number\month-%
\ifnum\day<10\relax0\fi\number\day}

\newenvironment{texte}{\rmfamily}{}

\newcommand{\fran}[0]{\selectlanguage{french}}

%\newcommand{\er}[0]{$^\mathrm{er}$}
%\newcommand{\re}[0]{$^\mathrm{re}$}
%\newcommand{\me}[0]{$^\mathrm{e}$}
\newcommand{\ee}[0]{$^\mathrm{e}$}
%\newcommand{\nd}[0]{$^\mathrm{nd}$}
%\newcommand{\nth}[0]{$^\mathrm{th}$}

\pagestyle{myheadings}
\markright{\sffamily\upshape\kern 8mm\datenum\hfill {\bfseries\normalsize Essai de Metaperlua\LaTeX}\kern 15mm\hfill Page }

\NoAutoSpaceBeforeFDP

\perlnewcommand{\levercoucher}[6]{
use DateTime::Event::Sunrise;
  my ($annee, $mois, $jour, $delta, $nb, $complet) = @_;
  my ($long, $lat) = qw/0.06 47.25/;
  my $lever = DateTime::Event::Sunrise->sunrise(longitude => $long
                          , latitude => $lat, iteration => 1);
  my $coucher = DateTime::Event::Sunrise->sunset(longitude => $long
                           , latitude => $lat, iteration => 1);
  #"Lever du soleil à " . $lever->next($day)->strftime("%H:%M:%S\n")
  #. "Coucher du soleil à " . $coucher->next($day)->strftime("%H:%M:%S\n");
  my $table;
  if ($complet) {
    $table = <<'TAB';
\begin{tabular}{|c|c|c|c|r|r|r|r|}
\hline
Jour & lever & midi & coucher & var.matin & var.soir & var.long & var.midi \\
\hline
TAB
  }
  else {
    $table = <<'TAB';
\begin{tabular}{|c|c|c|r|r|}
\hline
Jour & lever & coucher & var.matin & var.soir \\
\hline
TAB
  }
  my $day  = DateTime->new(year => $annee, month => $mois, day => $jour)->add(days => - $delta);
  my $matin = $lever->next($day);
  my $soir  = $coucher->next($day);
  my $s_matin = $matin->hour * 3600 + $matin->minute * 60 + $matin->second;
  my $s_soir  = $soir ->hour * 3600 + $soir ->minute * 60 + $soir ->second;
  my $s_midi  = ($s_matin + $s_soir) / 2;
  for (1..$nb) {
    my $s_matin_av = $s_matin;
    my $s_soir_av  = $s_soir;
    my $s_midi_av  = $s_midi;

    my $day  = $day->add(days => $delta);
    my $matin = $lever->next($day);
    my $soir  = $coucher->next($day);
    $s_matin = $matin->hour * 3600 + $matin->minute * 60 + $matin->second;
    $s_soir  = $soir ->hour * 3600 + $soir ->minute * 60 + $soir ->second;
    $s_midi  = ($s_matin + $s_soir) / 2;
    my $midi = $day->clone->add(seconds => $s_midi);

    if ($complet) {
      my $variations = sprintf("& %d & %d & %d & %d\\\\\n", $s_matin - $s_matin_av, $s_soir - $s_soir_av, $s_matin_av - $s_matin + $s_soir - $s_soir_av, $s_midi - $s_midi_av);
      $table .= $matin->strftime("%Y-%m-%d & %H:%M:%S\n") . '&' . $midi->strftime("%H:%M:%S") . '&' .  $soir->strftime("%H:%M:%S") . $variations;
    }
    else {
      my $variations = sprintf("& %d & %d\\\\\n", $s_matin - $s_matin_av, $s_soir - $s_soir_av);
      $table .= $matin->strftime("%Y-%m-%d & %H:%M:%S\n") . '&' .  $soir->strftime("%H:%M:%S") . $variations;
    }
  }
  $table .= "\\hline\n\\end{tabular}\n";
  return $table
}

% Affichage d'un nombre positif au format scientifique avec 3 décimales
% et calcul de la force d'attraction entre la Terre et un corps céleste
\directlua{
function affsci(n)
  local expo = math.floor(math.log10(n));
  local mant = n / 10 ^ expo;
  local facteur = 1000;
  mant = math.floor(mant * facteur) / facteur;
  return tex.sprint("$" .. tostring(mant) .. string.char(92) .. "times{}10^{" .. tostring(expo) .. "}$")
end
function force(masse, dist)
  local G = 6.673e-11;
  local mt = 5.973e+24;
  return affsci(G * mt * masse / (dist * dist));
end
}
\newcommand{\massesoleil}{1.989e+30}
\newcommand{\distsoleil}{1.49+12}
\newcommand{\masselune}{7.374e22}
\newcommand{\distlune}{3.844e+8}
\newcommand{\massejupiter}{1.898e+27}
\newcommand{\distjupiter}{778.3e+9}

\begin{document}
\begin{texte}
\tolerance=1000

\fran

\section{Exposé du problème}

Ce texte est une démonstration de l'outil \texttt{mpll}, ou 
\textit{Metaperlualatex}. C'est un document écrit en \LaTeX,
utilisant des fonctions Perl, des schémas MetaPost et (un peu)
des fonctions Lua.

Le présent document n'est pas pour autant un document artificiel
dont l'unique but est de démontrer l'utilisation de \textit{Metaperlualatex}.
C'est aussi une explication d'un phénomène que l'on peut observer au mois
de décembre. Pendant l'automne,
le soleil se lève de plus en plus tard et se couche de plus en
plus tôt. Cela n'étonne personne. De même, au printemps, le
soleil se lève de plus en plus tôt et se couche de plus
en plus tard. Aucun paradoxe là non plus.
Mais du 12~décembre au 2~janvier,
le soleil se lève de plus en plus tard et se couche
également de plus en plus tard. En d'autres termes,
le jour commence à gagner sur la nuit le
soir, tandis que la nuit continue à gagner sur le jour le matin,
comme on peut le voir dans le tableau suivant :

\vspace{2mm}
\levercoucher{2011}{12}{5}{7}{9}{0}
\vspace{2mm}

Cela dit, le texte n'est pas un exposé rigoureux de mécanique céleste,
une description quantitative précise détaillant le mouvement de la
Terre autour du soleil. C'est plutôt une mise au courant, une
présentation qualitative permettant de mieux comprendre les aspects
plutôt ignorés du mouvement de la Terre.

Pour un traitement plus rigoureux mais encore abordable, voir
``Histoire de l'Heure en France'', de Jacques Gapaillard, éditions
Vuibert -- ADAPT.

\section{Midi solaire et variation du lever et du coucher du soleil}

\subsection{Les diverses notions de midis}

Si l'on explique à une personne du XXI\ee{} siècle qu'il existe
deux notions de ``midi'' : le midi solaire, instant où le soleil
est au plus haut dans le ciel et où il passe au méridien et 
le midi moyen (ou plus précisément mais plus long, le
``midi du temps moyen''), que l'on peut repérer avec une montre ou une horloge,
si l'on explique que l'écart entre ces deux midis est variable,
cette personne du XXI\ee{} siècle répondra : «~Bien sûr, 
le midi moyen retarde d'une heure sur le midi solaire en hiver
et de deux heures en été.~» Éventuellement, si la personne réfléchit
un peu plus, elle précisera que l'écart varie en fonction du lieu
considéré. «~À Strasbourg, le midi solaire retarde d'une heure
et 31 minutes sur le midi moyen en hiver et de deux heures et
31 minutes en été, tandis qu'à Brest, le midi solaire retarde
de 42 minutes sur le midi moyen en hiver et d'une heure et 42 minutes
en été.~»

Cette personne associe la notion d'heure légale à l'expression ``midi moyen''
et la notion de midi du temps moyen à l'expression ``midi solaire''.
Si l'on tente de lui expliquer que, dans le cas de Strasbourg pendant l'hiver, l'écart
entre le midi solaire et le midi de l'heure légale varie
de 1 heure et 16 minutes à 1 heure et 46 minutes, alors la personne
ne comprendra plus.

À l'inverse, si l'on explique à une personne du XVIII\ee{} siècle que
le midi solaire oscille de part et d'autre du midi moyen et que
l'écart peut atteindre un quart d'heure, suivant une fonction appelée
``équation du temps'', il y a plus de chances que cette personne
acquiesce, pour peu que cette personne soit en contact avec des
horloges et des cadrans solaires. Cette personne ne comprendrait pas
la notion d'heure légale, qui n'existait pas à cette époque, mais elle
aurait compris l'irrégularité du passage du soleil au méridien. Il
était assez bien connu à cette époque que l'intervalle entre deux
midis solaires successifs n'était pas régulier. Cela dit, comme
l'expose J. Gapaillard dans son livre, même à cette époque il existait
des gens pour lesquels le mouvement apparent du soleil autour de la
Terre était régulier à la seconde près.

Dans la première partie, on admet que le midi solaire varie par rapport
au midi moyen et le document montre que cela permet d'expliquer le phénomène
de la variation du lever et du coucher du soleil au mois de décembre.
Dans la seconde partie, le document expliquera pour quelles
raisons le midi solaire varie par rapport au midi moyen.

\subsection{Exemple concret}

Voici par exemple, les heures de lever et de coucher du soleil, ainsi
que le midi solaire, à Saumur en mars 2011. J'ai choisi Saumur car la
longitude est de 0 degré et 4 minutes, donc le temps UTC correspond
quasiment au temps moyen local. Le tableau présente l'heure du lever
du soleil, le midi solaire, l'heure du coucher du soleil et les variations
en secondes de ces trois instants, ainsi que la variation de la longueur
du jour.

\vspace{2mm}
\levercoucher{2011}{3}{16}{1}{15}{1}
\vspace{2mm}

Prenons le 21 mars. Ce jour-là, la longueur du jour augmente de 205~s,
soit un peu plus de 3~mn. Également, le midi solaire recule de
12:07:53 à 12:07:35, soit 17 ou 18 secondes, compte tenu des erreurs
d'arrondi. L'augmentation de la longueur du jour se répercute pour
moitié sur la progression du lever du soleil (qui avance puisqu'on est
au printemps) et sur la progression du coucher du soleil (qui retarde).
À l'inverse, la variation du midi solaire se répercute à l'identique,
en durée et en sens, sur le lever du soleil et sur le coucher du
soleil. Donc, puisque le midi solaire avance de 17~s et puisque
le jour augmente de 205~s, le lever du soleil progresse de :

\[ -17 - \frac{205}{2} = -120 s \]

c'est-à-dire qu'il avance de 120~secondes,
passant de 06:03:07 à 06:01:07,
tandis que le coucher du soleil progresse de :

\[ -17 + \frac{205}{2} = 85 s \]

c'est-à-dire qu'il retarde de 85~secondes, passant de 18:12:39 à
18:14:04. Il y a donc une différence, mais qui passe inaperçue, car la
moitié de la variation du jour dépasse largement la variation du midi
solaire.

\vspace{5mm}

Maintenant, prenons le 24 décembre.

\vspace{2mm}
\levercoucher{2011}{12}{16}{1}{15}{1}
\vspace{2mm}

La durée du jour augmente de 6~secondes, tandis que le midi
solaire retarde de 30~secondes, passant de 11:58:25 à 11:58:55.
Le lever du soleil progresse donc de :

\[ +30 - \frac{6}{2} = +27 s \]

c'est-à-dire qu'il retarde de 27~secondes, passant
de 07:42:17 à 07:42:41, tandis que le
coucher du soleil progresse de :

\[ +30 + \frac{6}{2} = +33 s \]

c'est-à-dire que le coucher du soleil retarde de 33~secondes,
passant de 16:14:37 à 16:15:10.
Dans ce cas, la progression du midi solaire est nettement
prédominante par rapport à la moitié de la variation de la
durée du jour, donc le lever du soleil et le coucher
du soleil varient dans la même direction.

\vspace{5mm}

\subsection{Mea Culpa}

Dans le paragraphe précédent, j'ai commis une erreur de raisonnement.
J'ai assimilé le midi solaire au milieu de la journée, c'est-à-dire l'instant à mi-chemin entre
le lever et le coucher du soleil. C'est approximatif. Reprenons le
tableau du mois de mars :

\vspace{2mm}
\levercoucher{2011}{3}{20}{1}{3}{1}
\vspace{2mm}

Le 20~mars, le jour dure 12~h 9~mn et 32~s, soit 43~772~s.
Le 21~mars, il dure 12~h 12~mn et 57~s, soit 43~997~s
et le 22~mars, il dure 12~h 16~mn et 22~s, soit 44~182~s.
Cela ne veut pas dire que le 20~ mars à 23~h 59, la durée
du jour et de 43~772~s et que le 21~mars à 0~h 01, elle est
brusquement passée à 43~997~s. La durée varie de façon continue
et puisqu'elle varie de 205~s en 24~h, elle varie de 51~s en 6~h.
Cela donne :

\vspace{2mm}
\begin{tabular}{|c|c|c|}
\hline
20~mars, midi & 43~772 & 12~h 9~mn 32~s \\
\hline
20~mars, 18~h & 43~823 & 12~h 10~mn 23~s \\
\hline
21~mars, 6~h & 43~946 & 12~h 12~mn 6~s \\
\hline
21~mars, midi & 43~997 & 12~h 12~mn 57~s \\
\hline
21~mars, 18~h & 44~048 & 12~h 13~mn 48~s \\
\hline
22~mars, 6~h & 44~131 & 12~h 15~mn 31~s \\
\hline
22~mars, midi & 44~182 & 12~h 16~mn 22~s \\
\hline
\end{tabular}
\vspace{2mm}

Donc, le 21~mars, le lever du soleil se produit 6~h 6~mn et 3~s
avant le midi solaire et le coucher du soleil se produit
6~h 6~mn 54~s après le midi solaire. On en déduit que le
midi solaire a lieu à 12~h 7~mn et 10~s et non pas
à 12~h 7~mn 35~s comme dans le tableau ci-dessus.

L'écart entre le midi solaire et le milieu de la journée 
est proportionnel à la fois à la variation de la durée
du jour (maximale au moment des équinoxes) et à la durée du 
jour (maximale au solstice d'été). L'écart maximal se produit
aux alentours du 22~avril, lorsque la variation de la durée
du jour est encore voisine de la durée maximale et que la
durée du jour a eu le temps d'augmenter suffisamment.

\vspace{2mm}
\levercoucher{2011}{4}{21}{1}{3}{1}
\vspace{2mm}

À cette date à midi, la durée du jour est de 50~397~secondes (soit
14~heures à trois secondes près) et la variation de cette durée est de
193~secondes, à peine moins que les 205~secondes de l'équinoxe de
printemps. Donc, au lever du soleil, la durée est réduite de
$\frac{193 \times 7}{24} = 56 $ secondes et au coucher du soleil, elle
est augmentée d'autant. La durée du matin est $ 25~199 - 28 = 25~171 $
secondes et la durée du soir est $ 25~199 + 28 = 25~227 $ secondes.
Le midi solaire a donc lieu à 11~h 58~mn 38~s
alors que le milieu de la journée a lieu à 11~h 59~mn 6~s.

Cette erreur de calcul invalide-t-elle le raisonnement du
sous-chapitre précédent ? Non. Comme on peut le voir dans l'exemple
ci-dessus, cette erreur provoque un écart de 28~secondes au maximum
entre le midi solaire et le milieu de la journée. Cela reste encore
largement inférieur aux écarts affichés entre le midi moyen et le midi
solaire, écarts pouvant dépasser 15~minutes.

D'autre part, l'erreur de calcul sur la position du midi
solaire est quasiment nulle au moment du solstice d'hiver,
car la variation de la durée du jour est elle-même nulle.
Donc, dans le tableau pour la fin décembre et le début 
de janvier, la valeur indiquée pour le midi solaire
est quasiment exacte.


\section{Variation du midi solaire}

Le problème est donc maintenant de comprendre pourquoi
le midi solaire varie d'un jour à l'autre, parfois
de 30~secondes.

\subsection{Situation de départ}

Pour les mouvements de la Terre sur elle-même et autour
du soleil, adoptons les simplifications suivantes :

\begin{enumerate}

\item La durée de l'année est de 360~jours exactement et
non pas 365,25 et des poussières.

\item L'orbite de la Terre autour du soleil est un cercle parfait.

\item Le plan de l'orbite terrestre coïncide avec le plan de l'équateur.
En d'autres termes, l'axe de la rotation de la Terre sur elle-même est
parallèle à l'axe du mouvement orbital de la Terre autour du soleil.

\item On ne tient pas compte des mouvements à très longue échelle, comme
la variation de l'inclinaison du plan de l'équateur avec le plan de l'orbite
terrestre ou la précession des équinoxes.

\item On ne tient pas compte de l'influence des autres corps célestes :
la Lune, Jupiter etc. Avec sa masse \directlua{affsci(\massesoleil)}~kg,
le soleil exerce sur la Terre une force de \directlua{force(\massesoleil,\distsoleil)}~N
tandis que l'attraction de la Lune n'est que de \directlua{force(\masselune,\distlune)}~N
et celle de Jupiter de \directlua{force(\massejupiter,\distjupiter)}~N.
Donc on peut négliger l'influence des autres corps célestes et ne prendre
en considération que l'attraction du soleil.

\end{enumerate}

La première et les deux dernières hypothèses servent à simplifier les calculs
et resteront en vigueur pour toute la durée de la discussion. La deuxième
et la troisième simplifications seront supprimées en temps utile pour 
montrer les causes de la variabilité du midi solaire par rapport au 
midi moyen.

Avec les cinq simplifications en vigueur, la Terre tourne autour
du soleil en 360 jours à vitesse constante, donc elle parcourt un
arc d'un degré par jour. En adoptant temporairement un repère géocentrique
pointant vers les étoiles ``fixes'', cela crée donc un déplacement relatif
du soleil par rapport à ces étoiles ``fixes''. Pour compenser cet arc
d'un degré et pour que le soleil se retrouve au méridien, 
la rotation de la Terre autour de l'axe des pôles est
de 361~degrés en 24~heures. Donc, à J+90, elle a effectué 90~tours
et $\frac{1}{4}$, à J+180 elle a effectué 180~tours plus $\frac{1}{2}$
et à J+270 elle a effectué 270~tours plus $\frac{3}{4}$. Voir le
schéma ci-dessous, qui revient à un repère héliocentrique.

\begin{mplibcode}
beginfig(1);
ua = 100;
z0 = (ua,0);
z90 = (0,ua);
z180 = (-ua,0);
z270 = (0,-ua);
draw z0{up} .. z90 .. z180 .. z270 .. z0;

rayon = 5;
tiret = 2;
for i = 0 step 90 until 270 :
  % cercle de la Terre
  draw (x[i] + rayon, y[i]) .. (x[i], y[i] + rayon) .. (x[i] - rayon, y[i]) .. (x[i], y[i] - rayon) .. cycle;
  % repère pour le Soleil au méridien
  draw (x[i], y[i]) + (rayon, 0) rotated (i + 180) --  (x[i], y[i]) + (rayon + tiret, 0) rotated (i + 180);
endfor;

% soleil
draw (rayon, 0) .. (0, rayon) .. (-rayon, 0) .. (0, - rayon) .. (rayon, 0);

ecart = rayon + 5;
drawarrow ( x90 - ecart,  y90) {down} .. {right} ( x90,  y90 - ecart);
drawarrow (x180 - ecart, y180) .. (x180, y180 - ecart) .. (x180 + ecart, y180);
drawarrow (x270 - ecart, y270) .. (x270, y270 - ecart) .. (x270 + ecart, y270) .. (x270, y270 + ecart);

label.rt ("J+0",    (x0 + rayon,           y0));
label.top("J+90",   (x90,                  y90 + rayon));
label.lft("J+180",  (x180 - rayon - ecart, y180));
label.bot("J+270",  (x270,                 y270 - rayon - ecart));
label.bot("Soleil", (0,                    0 - rayon));

endfig;
\end{mplibcode}

Donc, lorsque la Terre effectue un mouvement circulaire et que le plan
de l'équateur coïncide avec le plan de l'orbite, le midi solaire coïncide
avec le midi moyen.

\subsection{Lois de Kepler}

La première loi de Kepler indique que l'orbite de la Terre
est une ellipse dont un foyer correspond au centre d'inertie
du système solaire, c'est-à-dire en pratique, au soleil.

La deuxième loi de Kepler indique que la vitesse de la Terre
sur son orbite varie de manière que la vitesse surfacique 
du vecteur soleil--Terre soit constante. Ou avec un peu 
plus de style, ``le rayon-vecteur balaie des surfaces égales
dans des temps égaux''.

Et la troisième loi de Kepler n'apporte rien à la présente
discussion.

Voyons ce que cela donne, avec une ellipse de forte 
excentricité (0,5) :

\begin{mplibcode}
beginfig(1);
ua = 100;
ub = 50 * sqrt(3);
z0 = (ua,0);
z61 = (0,ub);
z90 = (ua * cosd(116), ub * sind(116));
z180 = (-ua,0);
z270 = (x90, - y90);
z299 = (0,-ub);
draw z0{up} .. z61 .. z180 .. z299 .. z0;

rayon = 5;
tiret = 2;
for i = 0, 61, 90, 180, 270, 299 :
  % cercle de la Terre
  draw (x[i] + rayon, y[i]) .. (x[i], y[i] + rayon) .. (x[i] - rayon, y[i]) .. (x[i], y[i] - rayon) .. cycle;
endfor;

% soleil
xs = ua / 2;
ys = 0;
draw (xs + rayon, ys) .. (xs, ys + rayon) .. (xs - rayon, ys) .. (xs, ys - rayon) .. cycle;

label.rt  ("J+0",    (x0 + rayon,   y0));
label.urt ("J+61",   (x61,          y61 + rayon));
label.ulft("J+90",   (x90,          y90 + rayon));
label.lft ("J+180",  (x180 - rayon, y180));
label.llft("J+270",  (x270,         y270 - rayon));
label.lrt ("J+299",  (x299,         y299 - rayon));
label.lrt ("Soleil", (xs,           ys - rayon));

for i = 0 step 90 until 270 :
  draw (xs, ys) -- (x[i], y[i]);
endfor

endfig;
\end{mplibcode}

Comme le soleil se place à l'un des foyers de l'ellipse, à droite sur le
schéma, il faut déplacer les points J+90 et J+270 vers la gauche pour
que les surfaces (soleil, J+0, J+90), (soleil, J+90, J+180), (soleil, J+180,
J+270) et (soleil, J+270, J+0) soient égales en superficie. Et c'est donc
à J+61 et non pas J+90 que la Terre atteint l'extrémité du petit axe de l'orbite
et à J+299 au lieu de J+270 qu'elle atteint l'autre extrémité du petit axe.

\begin{mplibcode}
beginfig(1);
ua = 100;
ub = 50 * sqrt(3);
z0 = (ua,0);
z61 = (0,ub);
z90 = (ua * cosd(116), ub * sind(116));
z180 = (-ua,0);
z270 = (x90, - y90);
z299 = (0,-ub);
draw z0{up} .. z61 .. z180 .. z299 .. z0;

rayon = 5;
tiret = 2;
for i = 0, 61, 90, 180, 270, 299 :
  % cercle de la Terre
  draw (x[i] + rayon, y[i]) .. (x[i], y[i] + rayon) .. (x[i] - rayon, y[i]) .. (x[i], y[i] - rayon) .. cycle;
  % repère pour le Soleil au méridien
  draw (x[i], y[i]) + (rayon, 0) rotated (i + 180) --  (x[i], y[i]) + (rayon + tiret, 0) rotated (i + 180);
endfor;

% soleil
xs = ua / 2;
ys = 0;
draw (xs + rayon, ys) .. (xs, ys + rayon) .. (xs - rayon, ys) .. (xs, ys - rayon) .. cycle;

ecart = rayon + 5;
drawarrow ( x61 - ecart,  y61) {down} ..  ( (x61, y61) + (ecart, 0) rotated 241);
drawarrow ( x90 - ecart,  y90) {down} .. {right} ( x90,  y90 - ecart);
%drawarrow (x180 - ecart, y180) .. (x180, y180 - ecart) .. (x180 + ecart, y180);
drawarrow (x270 - ecart, y270) .. (x270, y270 - ecart) .. (x270 + ecart, y270) .. (x270, y270 + ecart);
%drawarrow (x299 - ecart, y299) .. (x299, y299 - ecart) .. (x299 + ecart, y299) .. (x299, y299 + ecart) .. ( (x299, y299) + (ecart, 0) rotated 109);

label.rt  ("J+0",    (x0 + rayon,           y0));
label.urt ("J+61",   (x61,                  y61 + rayon));
label.ulft("J+90",   (x90,                  y90 + rayon));
label.lft ("J+180",  (x180 - rayon - ecart, y180));
label.llft("J+270",  (x270,                 y270 - rayon - ecart));
label.lrt ("J+299",  (x299,                 y299 - rayon - ecart));
label.bot ("Soleil", (xs,                   ys - rayon));

endfig;
\end{mplibcode}

Le jour J+61 à midi moyen, la Terre a effectué 61~tours plus 61~degrés,
mais elle ne pointe pas vers le soleil. Il reste encore à effectuer
une rotation de 59~degrés pour pointer vers le soleil et arriver au midi
solaire. De même, le jour J+90 à midi moyen, la Terre a effectué 90~tours
plus 90~degrés et il reste à effectuer une rotation de 46~degrés pour
arriver au midi solaire. 

À l'inverse, au jour J+270, le midi solaire survient avant le midi moyen.
Le midi solaire se produit après 270~tours plus 224~degrés et le midi
moyen après 270~tours et 270~degrés, soit 46~degrés après le midi solaire.

\begin{mplibcode}
beginfig(1);

rayon = 20;
rayonf = 25;
pair midimoyen, midisolaire;

z61 = (0, 0);
draw (z61 + (rayon, 0)) .. (z61 + (0, rayon)) .. (z61 + (-rayon, 0)) .. (z61 + (0, -rayon)) .. cycle;
label.top("J+61", z61 + (0, rayon));

midimoyen   := z61 + (rayonf, 0) rotated 241;
midisolaire := z61 + (rayonf, 0) rotated 300;
draw (z61 + (rayon, 0) rotated 241) -- midimoyen;
draw (z61 + (rayon, 0) rotated 300) -- midisolaire;
label.llft("Midi moyen",   midimoyen);
label.lrt ("Midi solaire", midisolaire);
drawarrow (z61 - (rayonf, 0)) {down} .. midimoyen;
drawarrow midimoyen { dir 331 } .. { dir 30} midisolaire;

z90 = (100, 0);
draw (z90 + (rayon, 0)) .. (z90 + (0, rayon)) .. (z90 + (-rayon, 0)) .. (z90 + (0, -rayon)) .. cycle;
label.top("J+90", z90 + (0, rayon));
midimoyen   := z90 + (rayonf, 0) rotated 270;
midisolaire := z90 + (rayonf, 0) rotated 314;
draw (z90 + (rayon, 0) rotated 314) -- midisolaire;
draw (z90 + (rayon, 0) rotated 270) -- midimoyen;
label.bot ("Midi moyen",   midimoyen);
label.lrt ("Midi solaire", midisolaire);
drawarrow (z90 - (rayonf, 0)) {down} .. midimoyen;
drawarrow midimoyen { dir 0 } .. { dir 44 }  midisolaire;

z270 = (200, 0);
draw (z270 + (rayon, 0)) .. (z270 + (0, rayon)) .. (z270 + (-rayon, 0)) .. (z270 + (0, -rayon)) .. cycle;
label.top("J+270", z270 + (0, rayonf + 10));
midisolaire := z270 + (rayonf, 0) rotated 44;
midimoyen   := z270 + (rayonf, 0) rotated 90;
draw (z270 + (rayon, 0) rotated 44) -- midisolaire;
draw (z270 + (rayon, 0) rotated 90) -- midimoyen;
label.top("Midi moyen",   midimoyen);
label.rt ("Midi solaire", midisolaire);
%drawarrow (z270 - (rayonf, 0)) {down} .. midisolaire;
drawarrow (z270 - (rayonf, 0)) {down} .. (z270 - (0, rayonf)) .. (z270 + (rayonf, 0)) .. midisolaire;
drawarrow midisolaire { dir 134 } .. { dir 180}  midimoyen;

z299 = (350, 0);
draw (z299 + (rayon, 0)) .. (z299 + (0, rayon)) .. (z299 + (-rayon, 0)) .. (z299 + (0, -rayon)) .. cycle;
label.top("J+299", z299 + (0, rayonf + 10));

midisolaire := z299 + (rayonf, 0) rotated 63;
midimoyen   := z299 + (rayonf, 0) rotated 119;
draw (z299 + (rayon, 0) rotated 63) -- midisolaire;
draw (z299 + (rayon, 0) rotated 119) -- midimoyen;
label.urt ("Midi solaire", midisolaire);
label.ulft("Midi moyen",   midimoyen);
%drawarrow (z299 - (rayonf, 0)) {down} .. (z299 - (0, rayonf)) .. (z99 + (0, rayonf)) .. midisolaire;
drawarrow (z299 - (rayonf, 0)) {down} .. (z299 - (0, rayonf)) .. midisolaire;
drawarrow midisolaire { dir 153 } .. { dir 209 } midimoyen;

endfig;
\end{mplibcode}

Dans ces exemples, les écarts de 46 ou 59~degrés correspondent
à un écart de temps de 3 ou 4~heures. C'est largement supérieur
aux écarts observés par la Terre réelle. Mais également, l'orbite
de la Terre a une excentricité de 0,02, alors que la discussion
ci-dessus repose sur une excentricité de 0,5.

\subsection{Influence de l'inclinaison de l'orbite}

On remet en vigueur la deuxième simplification et on enlève la
troisième. La discussion est alors un peu plus difficile à suivre,
car cela impose un raisonnement en dimension~3, peu commode
à représenter sur une feuille de papier en dimension~2.

On part de la situation où les quatre simplifications sont
en vigueur, puis on fait pivoter le plan de l'orbite terrestre
\textit{sans faire pivoter le plan de l'équateur ou l'axe des pôles},
puis on projette orthogonalement la nouvelle orbite sur le plan de l'ancienne
orbite, \textit{toujours sans faire pivoter le plan de l'équateur}.

\begin{mplibcode}
beginfig(1);

ua = 100;
rayon = 5;
tiret = 2;

% Terre vue de profil
def terreprofil(expr z) =
fill z + (rayon, 0) .. z + (0, rayon) .. z - (rayon, 0) .. z - (0, rayon) .. cycle;
draw z - (rayon + tiret, 0) -- z + (rayon + tiret, 0);
enddef;

% transformation vue de profil
xprof = -200;
draw (xprof, ua) -- (xprof, -ua);
draw (xprof, 0) + (0, ua) rotated 60 -- (xprof, 0) + (0, -ua) rotated 60;
terreprofil( (xprof, ua ) );
terreprofil( (xprof, 0 ) + (0, ua) rotated 60 );
terreprofil( (xprof, ua / 2 ) );
terreprofil( (xprof, - ua ) );
terreprofil( (xprof, 0 ) - (0, ua) rotated 60 );
terreprofil( (xprof, - ua / 2 ) );

drawarrow  (xprof - rayon - 3 * tiret, ua) {left} .. {dir 240} ((xprof, 0 ) + (0 + rayon + 3 * tiret, ua + tiret) rotated 60);
drawarrow  (xprof + rayon + 3 * tiret, 0 ) + (0, ua) rotated 60 --  (xprof - rayon - 3 * tiret, ua / 2);
drawarrow  (xprof + rayon + 3 * tiret, - ua) {right} .. {dir 60} ((xprof, 0 ) - (0 + rayon + 3 * tiret, ua + tiret) rotated 60);
drawarrow  (xprof - rayon - 3 * tiret, 0 ) - (0, ua) rotated 60 --  (xprof + rayon + 3 * tiret, - ua / 2);

% transformation vue de face
z0 = (ua,0);
z90 = (0,ua);
z180 = (-ua,0);
z270 = (0,-ua);

pair zf[];
path orbite;
zf90 = (0, ua / 2);
zf270 = (0, - ua / 2);
orbite = z0{up} .. z90 .. z180 .. z270 .. z0;
draw orbite;
draw orbite yscaled 0.5;

for i = 0 step 90 until 270 :
  % cercle de la Terre
  draw (x[i] + rayon, y[i]) .. (x[i], y[i] + rayon) .. (x[i] - rayon, y[i]) .. (x[i], y[i] - rayon) .. cycle;
  % repère pour le Soleil au méridien
  draw (x[i], y[i]) + (rayon, 0) rotated (i + 180) --  (x[i], y[i]) + (rayon + tiret, 0) rotated (i + 180);
endfor;

% soleil
draw (rayon, 0) .. (0, rayon) .. (-rayon, 0) .. (0, - rayon) .. (rayon, 0);

% Nouveau J+90
draw ((rayon, 0) .. (0, rayon) .. (- rayon, 0) .. (0, - rayon) .. cycle) shifted zf90;
draw ((0, - rayon) -- (0, - rayon - tiret)) shifted (x90, y90 / 2);

drawarrow (z90 - (0, 2 * rayon)) -- (zf90 + (0, 2 * rayon));

% Nouveau J+270
draw ((rayon, 0) .. (0, rayon) .. (- rayon, 0) .. (0, - rayon) .. cycle) shifted zf270;
draw ((0, + rayon) -- (0, + rayon + tiret)) shifted zf270;

drawarrow (z270 + (0, 2 * rayon)) -- (zf270 - (0, 2 * rayon));

ecart := rayon + 5;
drawarrow ( x90 - ecart,  y90) {down} .. {right} ( x90,  y90 - ecart);
drawarrow (x180 - ecart, y180) .. (x180, y180 - ecart) .. (x180 + ecart, y180);
drawarrow (x270 - ecart, y270) .. (x270, y270 - ecart) .. (x270 + ecart, y270) .. (x270, y270 + ecart);

label.rt ("J+0",    (x0 + rayon,           y0));
label.top("J+90",   (x90,                  y90 + rayon));
label.lft("J+180",  (x180 - rayon - ecart, y180));
label.bot("J+270",  (x270,                 y270 - rayon - ecart));
label.bot("Soleil", (0,                    0 - rayon));

endfig;
\end{mplibcode}

Cela se traduit au final par un rétrécissement du cercle sur un
axe, ce qui donne une ellipse. Tout ceci, sans changer la vitesse
de rotation de la Terre sur elle-même, puisque l'axe des pôles
est resté constamment parallèle à lui-même.

Compte tenu de l'inclinaison exagérée de 60~degrés, le cercle
est rétréci de moitié. Pour l'inclinaison réelle de 23~degrés,
le rétrécissement du cercle aurait été de  8~\%.

Prenons maintenant en compte les jours intermédiaires, J+45,
J+135, J+225 et J+315.

\begin{mplibcode}
beginfig(1);

ua = 100;
rayon = 5;
tiret = 2;
% transformation vue de face
z0   = ( ua,  0);
z90  = (  0, ua);
z180 = (-ua,  0);
z270 = (  0,-ua);

path orbite;
orbite = z0{up} .. z90 .. z180 .. z270 .. z0;

z45  = point 0.5 of orbite;
z135 = point 1.5 of orbite;
z225 = point 2.5 of orbite;
z315 = point 3.5 of orbite;

pair zf[];
zf90 = (0, ua / 2);
xf90 = 0;
yf90 = ua / 2;
zf270 = (0, - ua / 2);
xf270 = 0;
yf270 = - ua / 2;
zf45  = z45  yscaled 0.5;
zf135 = z135 yscaled 0.5;
zf225 = z225 yscaled 0.5;
zf315 = z315 yscaled 0.5;

draw orbite;
draw orbite yscaled 0.5;

for i = 0 step 45 until 315 :
  % cercle de la Terre
  draw (x[i] + rayon, y[i]) .. (x[i], y[i] + rayon) .. (x[i] - rayon, y[i]) .. (x[i], y[i] - rayon) .. cycle;
  % repère pour le Soleil au méridien
  draw (x[i], y[i]) + (rayon, 0) rotated (i + 180) --  (x[i], y[i]) + (rayon + tiret, 0) rotated (i + 180);
endfor;

% soleil
draw (rayon, 0) .. (0, rayon) .. (-rayon, 0) .. (0, - rayon) .. (rayon, 0);

% Nouveau J+45
draw ((rayon, 0) .. (0, rayon) .. (- rayon, 0) .. (0, - rayon) .. cycle) shifted zf45;
draw ((- rayon, 0) -- (- rayon - tiret, 0)) rotated 45 shifted (x45, y45 / 2);

drawarrow (z45 - (0, 2 * rayon)) -- (zf45 + (0, 2 * rayon));

% Nouveau J+90
draw ((rayon, 0) .. (0, rayon) .. (- rayon, 0) .. (0, - rayon) .. cycle) shifted zf90;
draw ((0, - rayon) -- (0, - rayon - tiret)) shifted (x90, y90 / 2);

drawarrow (z90 - (0, 2 * rayon)) -- (zf90 + (0, 2 * rayon));

% Nouveau J+135
draw ((rayon, 0) .. (0, rayon) .. (- rayon, 0) .. (0, - rayon) .. cycle) shifted zf135;
draw ((- rayon, 0) -- (- rayon - tiret, 0)) rotated 135 shifted (x135, y135 / 2);

drawarrow (z135 - (0, 2 * rayon)) -- (zf135 + (0, 2 * rayon));

% Nouveau J+225
draw ((rayon, 0) .. (0, rayon) .. (- rayon, 0) .. (0, - rayon) .. cycle) shifted zf225;
draw ((- rayon, 0) -- (- rayon - tiret, 0)) rotated 225 shifted (x225, y225 / 2);

drawarrow (z225 + (0, 2 * rayon)) -- (zf225 - (0, 2 * rayon));

% Nouveau J+270
draw ((rayon, 0) .. (0, rayon) .. (- rayon, 0) .. (0, - rayon) .. cycle) shifted zf270;
draw ((0, + rayon) -- (0, + rayon + tiret)) shifted zf270;

drawarrow (z270 + (0, 2 * rayon)) -- (zf270 - (0, 2 * rayon));

% Nouveau J+315
draw ((rayon, 0) .. (0, rayon) .. (- rayon, 0) .. (0, - rayon) .. cycle) shifted zf315;
draw ((- rayon, 0) -- (- rayon - tiret, 0)) rotated 315 shifted (x315, y315 / 2);

drawarrow (z315 + (0, 2 * rayon)) -- (zf315 - (0, 2 * rayon));

ecart := rayon + 5;
%drawarrow ( x90  - ecart,  y90)  {down} .. {right} ( x90,   y90  - ecart);
%drawarrow ( xf90 - ecart,  yf90) {down} .. {right} ( xf90,  yf90 - ecart);
%drawarrow (x180 - ecart, y180) .. (x180, y180 - ecart) .. (x180 + ecart, y180);
%drawarrow (x270  - ecart, y270 ) .. (x270 , y270  - ecart) .. (x270  + ecart, y270 ) .. (x270 , y270  + ecart);
%drawarrow (xf270 - ecart, yf270) .. (xf270, yf270 - ecart) .. (xf270 + ecart, yf270) .. (xf270, yf270 + ecart);

label.rt ("J+0",    (x0 + rayon,           y0));
label.rt ("J+45",   (x45 + rayon,          y45));
label.top("J+90",   (x90,                  y90 + rayon));
label.lft("J+135",  (x135 - rayon - ecart, y135));
label.lft("J+180",  (x180 - rayon - ecart, y180));
label.lft("J+225",  (x225 - rayon - ecart, y225));
label.bot("J+270",  (x270,                 y270 - rayon - ecart));
label.rt ("J+315",  (x315 + rayon,         y315));
label.bot("Soleil", (0,                    0 - rayon));

endfig;
\end{mplibcode}

Avec cette transformation, on constate que le midi moyen et le midi
solaire coïncident pour J+90, J+180 et J+270. Mais ce n'est pas
le cas entre ces différents jours, notamment pour J+45,
J+135, J+225 et J+315.


\begin{mplibcode}
beginfig(1);

rayon = 20;
rayonf = 25;
pair midimoyen, midisolaire;

z45 = (0, 0);
draw (z45 + (rayon, 0)) .. (z45 + (0, rayon)) .. (z45 + (-rayon, 0)) .. (z45 + (0, -rayon)) .. cycle;
label.top("J+45", z45 + (0, rayon));
midimoyen   := z45 + (rayonf, 0) rotated 225;
midisolaire := z45 + (rayonf, 0) rotated 207;
draw (z45 + (rayon, 0) rotated 225) -- midimoyen;
draw (z45 + (rayon, 0) rotated 207) -- midisolaire;
label.bot ("Midi moyen",   midimoyen);
label.lft ("Midi solaire", midisolaire);
drawarrow (z45 - (rayonf, 0)) {down} .. midisolaire;
drawarrow midisolaire { dir 297 } .. { dir 315} midimoyen;

z135 = (70, 0);
draw (z135 + (rayon, 0)) .. (z135 + (0, rayon)) .. (z135 + (-rayon, 0)) .. (z135 + (0, -rayon)) .. cycle;
label.top("J+135", z135 + (0, rayon));
midimoyen   := z135 + (rayonf, 0) rotated 315;
midisolaire := z135 + (rayonf, 0) rotated 333;
draw (z135 + (rayon, 0) rotated 315) -- midimoyen;
draw (z135 + (rayon, 0) rotated 333) -- midisolaire;
label.lrt("Midi moyen",   midimoyen);
label.rt ("Midi solaire", midisolaire);
drawarrow (z135 - (rayonf, 0)) {down} .. midimoyen;
drawarrow midimoyen { dir 45 } .. { dir 63}  midisolaire;

z225 = (180, 0);
draw (z225 + (rayon, 0)) .. (z225 + (0, rayon)) .. (z225 + (-rayon, 0)) .. (z225 + (0, -rayon)) .. cycle;
label.top("J+225", z225 + (0, rayon));
midisolaire := z225 + (rayonf, 0) rotated  27;
midimoyen   := z225 + (rayonf, 0) rotated  45;
draw (z225 + (rayon, 0) rotated  27) -- midisolaire;
draw (z225 + (rayon, 0) rotated  45) -- midimoyen;
label.rt  ("Midi solaire", midisolaire);
label.urt ("Midi moyen",   midimoyen);
drawarrow (z225 - (rayonf, 0)) {down} .. midisolaire;
drawarrow midisolaire { dir 217 } .. { dir 135} midimoyen;

z315 = (350, 0);
draw (z315 + (rayon, 0)) .. (z315 + (0, rayon)) .. (z315 + (-rayon, 0)) .. (z315 + (0, -rayon)) .. cycle;
label.top("J+315", z315 + (0, rayonf));
midimoyen   := z315 + (rayonf, 0) rotated 135;
midisolaire := z315 + (rayonf, 0) rotated 153;
draw (z315 + (rayon, 0) rotated 135) -- midimoyen;
draw (z315 + (rayon, 0) rotated 153) -- midisolaire;
label.ulft("Midi moyen",   midimoyen);
label.lft ("Midi solaire", midisolaire);
drawarrow (z315 - (rayonf, 0)) {down} .. (z315 - (0, rayonf)) .. (z315 + (rayonf, 0)) .. midimoyen;
drawarrow midimoyen { dir 225 } .. { dir 243}  midisolaire;

endfig;
\end{mplibcode}

On voit que le midi solaire précède le midi moyen de J+0 à J+90 puis de nouveau
de J+180 à J+270, tandis que le midi moyen précède le midi solaire de
J+90 à J+180 et de nouveau de J+270 à J+360. Au lieu d'une variation périodique
de période égale à une année, on a maintenant une variation périodique
de période égale à la moitié d'une année.

\subsection{Au final}

Avec les valeurs prises, excentricité à 0,5 et inclinaison à
60~degrés, la prise en compte simultanée des deux phénomènes est
compliquée. En revanche, avec les valeurs réelles de la Terre,
excentricité à moins de 0,02 et inclinaison à 23~degrés, on peut
assimiler les courbes de variation à des sinusoïdales de faible
amplitude et les additionner. Attention, les jours J+0 pour les deux
phénomènes ne coïncident pas. Pour l'excentricité de l'orbite, J+0
correspond au passage de la Terre au périhélie, soit le
3~janvier. Pour l'inclinaison de l'orbite, J+0 correspond à l'un des
équinoxes, le 21~mars ou le 21~septembre environ.

La courbe de variation est donnée dans ``Histoire de l'Heure en France'', 
de Jacques Gapaillard, éditions Vuibert -- ADAPT. L'équation du temps (écart
entre le midi solaire et le midi moyen) varie entre  un retard de 14~mn
15~s du midi solaire par rapport au midi moyen et une avance de 16~mn
et 30~s du midi solaire par rapport au midi moyen.

\section{Annexe}

Ce texte est diffusé sous la licence \textit{Creative Commons},
Paternité, pas d'utilisation commerciale 3.0 France.
Les programmes associés sont diffusés sous license double GPL + \textit{Artistic}.
Copyright (c) 2012, Jean Forget.

\end{texte}
\end{document}
