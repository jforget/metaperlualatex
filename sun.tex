\documentclass[a4paper]{article}
%\usepackage[english,frenchb,german,italian,latin]{babel}
\usepackage[utf8]{luainputenc}
\usepackage{luaotfload}
%\usepackage{graphicx}
%\usepackage{textcomp}
%\usepackage[dvips]{hyperref}
\usepackage{breakurl}
\usepackage{perltex}
%\usepackage{luamplib}
\usepackage{luatexko-mplib}

% marge haute 25.4 - 12 = 13.4 mm
\topmargin=-12mm
\headheight=0mm
\headsep=4mm
% marge basse 297 - (13.4 + 267 + 4) = 12.6mm
\textheight=267mm

% marge gauche 25,4mm,
% marge droite 210 - (25,4 + 173) = 11,6mm
\oddsidemargin=0mm
\marginparsep=0mm
\textwidth=173mm
\columnsep=8mm

%&Latex

\renewcommand{\rmdefault}[0]{ppl}
\renewcommand{\sfdefault}[0]{phv}
\renewcommand{\ttdefault}[0]{pcr}

\newcommand{\datenum}{\number\year-%
\ifnum\month<10\relax0\fi\number\month-%
\ifnum\day<10\relax0\fi\number\day}

\newenvironment{texte}{\rmfamily}{}

%\newcommand{\fran}[0]{\selectlanguage{french}}
%\newcommand{\angl}[0]{\selectlanguage{english}}

%\newcommand{\er}[0]{$^\mathrm{er}$}
%\newcommand{\re}[0]{$^\mathrm{re}$}
%\newcommand{\me}[0]{$^\mathrm{e}$}
\newcommand{\ee}[0]{$^\mathrm{e}$}
%\newcommand{\nd}[0]{$^\mathrm{nd}$}
%\newcommand{\nth}[0]{$^\mathrm{th}$}

\pagestyle{myheadings}
\markright{\sffamily\upshape\kern 8mm\datenum\hfill {\bfseries\normalsize Trial of Metaperlua\LaTeX}\kern 15mm\hfill Page }

%\NoAutoSpaceBeforeFDP

\newcommand{\everisehhmmss}[0]{08:15:00}
\newcommand{\evenoonhhmmss}[0]{12:00:00}
\newcommand{\evesethhmmss}[0]{18:15:00}
\newcommand{\dayrisehhmmss}[0]{08:15:00}
\newcommand{\daynoonhhmmss}[0]{12:00:00}
\newcommand{\daysethhmmss}[0]{18:15:00}
\newcommand{\deltalight}[0]{0}
\newcommand{\deltarise}[0]{0}
\newcommand{\deltanoon}[0]{0}
\newcommand{\deltaset}[0]{0}
\newcommand{\formularise}[0]{\[ \deltanoon - \frac{\deltalight}{2} = \deltarise \]}
\newcommand{\formulaset}[0]{\[ \deltanoon + \frac{\deltalight}{2} = \deltaset \]}

\perlnewcommand{\riseandset}[5]{
use v5.10;
use strict;
use warnings;
use DateTime::Event::Sunrise;
  my ($day_begin, $delta, $nb, $full, $show_day) = @_;
  my ($year, $month, $day)  = unpack("a4 a2 a2", $day_begin);
  my ($long, $lat) = qw/0 51.47/; # Greenwhich
  # my ($long, $lat) = qw/-5.53 50.08/; # Penzance
  my $lever = DateTime::Event::Sunrise->sunrise(longitude => $long
                          , latitude => $lat, iteration => 1);
  my $coucher = DateTime::Event::Sunrise->sunset(longitude => $long
                           , latitude => $lat, iteration => 1);
  #"Lever du soleil à " . $lever->next($day)->strftime("%H:%M:%S\n")
  #. "Coucher du soleil à " . $coucher->next($day)->strftime("%H:%M:%S\n");

  my ($aaaa, $mm, $jj) = unpack("a4 a2 a2", $show_day);
  print "year $aaaa, month $mm, day $jj\n";
  my $show_dt = DateTime->new(year => $aaaa, month => $mm, day => $jj, locale => 'en');
  my $eve  = $show_dt->clone->add(days => -1);
  print $eve->strftime("%Y-%m-%d%n");

  my $dayrisehhmmss = '';
  my $everisehhmmss = '';
  my $evenoonhhmmss = '';
  my $daynoonhhmmss = '';
  my $evesethhmmss = '';
  my $daysethhmmss = '';
  my $deltalight = 0;
  my $deltarise  = 0;
  my $deltanoon  = 0;
  my $deltaset   = 0;
  my $show_day_found = 0;

  my $table;
  if ($full) {
    $table = <<'TAB';
\begin{tabular}{|c|c|c|c|r|r|r|r|}
\hline
Day & rise & noon & set & var.morn & var.even & var.length & var.noon \\ 
\hline
TAB
  }
  else {
    $table = <<'TAB';
\begin{tabular}{|c|c|c|r|r|}
\hline
Day & rise & set & var.morn & var.even \\
\hline
TAB
  }
  my $day  = DateTime->new(year => $year, month => $month, day => $day)->add(days => - $delta);
  my $matin = $lever->next($day);
  my $soir  = $coucher->next($day);
  my $s_matin = $matin->hour * 3600 + $matin->minute * 60 + $matin->second;
  my $s_soir  = $soir ->hour * 3600 + $soir ->minute * 60 + $soir ->second;
  my $s_midi  = ($s_matin + $s_soir) / 2;
  for (1..$nb) {
    my $s_matin_av = $s_matin;
    my $s_soir_av  = $s_soir;
    my $s_midi_av  = $s_midi;

    my $day  = $day->add(days => $delta);
    my $matin = $lever->next($day);
    my $soir  = $coucher->next($day);
    $s_matin = $matin->hour * 3600 + $matin->minute * 60 + $matin->second;
    $s_soir  = $soir ->hour * 3600 + $soir ->minute * 60 + $soir ->second;
    $s_midi  = ($s_matin + $s_soir) / 2;
    my $midi = $day->clone->add(seconds => $s_midi);

    if ($full) {
      my $variations = sprintf("& %d & %d & %d & %d\\\\\n", $s_matin - $s_matin_av, $s_soir - $s_soir_av, $s_matin_av - $s_matin + $s_soir - $s_soir_av, $s_midi - $s_midi_av);
      $table .= $matin->strftime("%Y-%m-%d & %H:%M:%S\n") . '&' . $midi->strftime("%H:%M:%S") . '&' .  $soir->strftime("%H:%M:%S") . $variations;
    }
    else {
      my $variations = sprintf("& %d & %d\\\\\n", $s_matin - $s_matin_av, $s_soir - $s_soir_av);
      $table .= $matin->strftime("%Y-%m-%d & %H:%M:%S\n") . '&' .  $soir->strftime("%H:%M:%S") . $variations;
    }

    if ($day->strftime("%Y%m%d") eq $show_day) {
      $dayrisehhmmss = $matin->strftime("%H:%M:%S");
      $daynoonhhmmss = $midi->strftime("%H:%M:%S");
      $daysethhmmss  = $soir->strftime("%H:%M:%S");
      $deltalight    = $s_matin_av - $s_matin + $s_soir - $s_soir_av;
      $deltarise     = $s_matin - $s_matin_av;
      $deltanoon     = $s_midi  - $s_midi_av;
      $deltaset      = $s_soir  - $s_soir_av;
      #print $dayrisehhmmss, "\n";
      $show_day_found = 1;
    }
    elsif ($show_day_found == 0) {
      $everisehhmmss = $matin->strftime("%H:%M:%S");
      $evenoonhhmmss = $midi->strftime("%H:%M:%S");
      $evesethhmmss  = $soir->strftime("%H:%M:%S");
    }

  }
  my ($sgn_rise, $sgn_noon, $sgn_set) = ('', '', '');
  if ($deltarise > 0) {
    $sgn_rise = '+';
  }
  if ($deltanoon > 0) {
    $sgn_noon = '+';
  }
  if ($deltaset > 0) {
    $sgn_set = '+';
  }
  $table .= <<"EOF";
\\hline\n\\end{tabular}
\\renewcommand{\\dayrisehhmmss}[0]{$dayrisehhmmss}
\\renewcommand{\\everisehhmmss}[0]{$everisehhmmss}
\\renewcommand{\\evenoonhhmmss}[0]{$evenoonhhmmss}
\\renewcommand{\\daynoonhhmmss}[0]{$daynoonhhmmss}
\\renewcommand{\\evesethhmmss}[0]{$evesethhmmss}
\\renewcommand{\\daysethhmmss}[0]{$daysethhmmss}
\\renewcommand{\\deltalight}[0]{$deltalight}
\\renewcommand{\\deltarise}[0]{$deltarise}
\\renewcommand{\\deltanoon}[0]{$deltanoon}
\\renewcommand{\\deltaset}[0]{$deltaset}
\\renewcommand{\\formularise}[0]{\\[ $sgn_noon$deltanoon - \\frac{$deltalight}{2} = $sgn_rise$deltarise \\]}
\\renewcommand{\\formulaset}[0]{\\[ $sgn_noon$deltanoon + \\frac{$deltalight}{2} = $sgn_set$deltaset \\]}
EOF
  return $table
}

% Display a positive number in scientific mode with three digits after the decimal point
% and computation of the gravitational force between Earth and another celestial body
\directlua{
function affsci(n)
  local expo = math.floor(math.log(n) / math.log(10));
  local mant = n / 10 ^ expo;
  local factor = 1000;
  mant = math.floor(mant * factor) / factor;
  return tex.sprint("$" .. tostring(mant) .. string.char(92) .. "times{}10^{" .. tostring(expo) .. "}$")
end
function force(mass, dist)
  local G         = 6.673e-11;
  local earthmass = 5.973e+24;
  return affsci(G * earthmass * mass / (dist * dist));
end
}
\newcommand{\solarmass}{1.989e+30}
\newcommand{\solardist}{1.49+12}
\newcommand{\lunarmass}{7.374e22}
\newcommand{\lunardist}{3.844e+8}
\newcommand{\jupitermass}{1.898e+27}
\newcommand{\jupiterdist}{778.3e+9}

\begin{document}
\begin{texte}
\tolerance=1000

%\angl

\section{What Is The Problem}

This text is a showcase for the \texttt{mpll} tool, or
\textit{Metaperlualatex} tool. This is a document written in \LaTeX,
and using Perl functions, MetaPost drawings
and Lua functions (a few).

Yet, this document is not an artificial document whose
only purpose is to pack significant \textit{Metaperlualatex} features in a
meaningless text. It is also an explanation 
of a real phenomenon which can be observed
in December (in temperate areas in the Northern hemisphere).
During the fall, the sun rises later and later
and sets sooner and sooner. No surprise there.
During the spring, the sun rises earlier and earlier
and sets later and later.
No surprise there either. But from
14th December until 1st January
the sun rises later and later and it also sets later 
and later. In other words, the day gradually replaces
the night during the evening, while the night
still gradually replaces the day during the morning,
as can be seen in the table below.

\vspace{2mm}
\riseandset{20111205}{7}{9}{0}{20111224}
\vspace{2mm}

On the other side, this document is not a scientific publication
with strict computations of celestial movements.
It is rather an explanation for an intuitive
understanding of some mostly unknown aspects
of Earth's movement.

For a more precise explanation, yet still readable by a layman,
you can refer to the French-written book
``Histoire de l'Heure en France'', by Jacques Gapaillard, 
published by Vuibert -- ADAPT.

As said above, the document applies to temperate areas in
the Northern hemisphere. Around the equator, the variation 
of daylight is not sufficient for the reasoning to be valid.
In the polar areas, the continous night in winter and the continuous
daylight in summer play havoc with the reasoning. And in temperate
areas of the Southern hemisphere, it might be sufficient to
shift all dates by six months, so the winter falls in June and the
spring falls in September.
The algorithm used to compute the sunrise and the sunset is
Paul Schlyter's algorithm.

\section{Solar Noon And The Variations Of Sunrise And Sunset}

\subsection{The Various Types Of Noon}

%\vspace{2mm}
%\riseandset{20120210}{1}{10}{1}{20111224}
%
%\riseandset{20121101}{1}{10}{1}{20111224}
%\vspace{2mm}

If you explain to a XXIth Century person that there are
two notions of ``noon'', the astronomical noon, which is
the instant where the sun is at its highest in the 
sky and is aligned with the local meridian, and the
mean noon (or rather, ``mean time noon''), which can
be obtained with a watch or a clock, if you explain that
the offset between these two instants varies from a
day to the other, this XXIth century person, if leaving
in Great Britain for example, will answer
``Of course, during the summer the astronomical noon is
one hour behind the mean noon, while they are the same
during winter.'' After a little thinking, this person
will add that this also depends on the place you live
in: ``This is for Kent, although in Cornwall, the
astronomical noon is half-an-hour behind
the mean noon in winter and one hour and a half minutes behind
the mean noon in summer.''

This person confuses the ``legal hour'' with the ``mean time''
and the ``mean noon'' with the ``astronomical noon''.
If we attempt to explain him that, for Kent in winter,
the difference between ``astronomical noon'' and ``mean noon''
may vary from 14 minutes in one direction
and 17 minutes in the other, this person
will no longer understand.

On the other side, if you explain to a XVIIIth century
person that the astronomical noon oscillates on both
sides of the mean noon and that this gap may reach
one quarter of an hour, it is more likely
that this person will approve, especially if he works
with clocks or sundials. This person would not understand
the notion of ``legal time'', because it did not exist
at the time, but he would understand the irregularities
of the crossing of the meridian by the sun.
In this time, it was well-known that the interval
between two successive times when the sun crosses
the meridian was not a constant. Yet, as J. Gapaillard
explains in his book, even in this time, there were many people
who erroneously thought that the apparent movement of
the sun around the Earth was constant within one second.

In the first part, we posit, without explanation, that
the astronomical noon varies with respect to the 
mean noon and we explain how this results in the 
paradoxical variation of sunrise and sunset in December.
In the second part, the document explains why there
exists a variation between the astronomical noon
and the mean noon.

\subsection{Real Life Example}

The table below give the sunrise and sunset hours, as well
as the astronomical noon, in Greenwich in March 2011.
Using Greenwich allows to equate the UTC time with
the mean time (the formerly well-known GMT), within a
few seconds. The table gives the time of sunrise,
of sunset, of noon and the variations (in seconds)
with the previous day. Also shown is the variation
of the day length.

\vspace{2mm}
\riseandset{20110316}{1}{15}{1}{20110321}
\vspace{2mm}

Let us consider 21st March. On this date, the daylight
increases by \deltalight{} seconds, that is, a bit less
than 4 minutes. The solar noon switch
from \evenoonhhmmss{} to \daynoonhhmmss{} so it shifts
by \deltanoon{} seconds. One half of the daylight increase
applies to the sunrise (which comes earlier) and the other half to sunset
(which comes later). On the other side, the variation of 
solar noon applies in full to both sunrise and sunset, and
with the same direction. So, since the solar noon shifts
by \deltanoon{} seconds and the daylight increases by
\deltalight{} seconds, sunrise moves by

\formularise{}

that is, it moves backward from \everisehhmmss{} to \dayrisehhmmss{}
while sunset moves by

\formulaset{}

that is, it moves forward from \evesethhmmss{} to \daysethhmmss{}.
There is a difference, but most persons do not notice,
since the daylight variation is much greater than the
solar noon shift.

\vspace{5mm}

Now, let us consider 23th December.

\vspace{2mm}
\riseandset{20111216}{1}{15}{1}{20111223}
\vspace{2mm}

The daylight increases by \deltalight{} second,
while the solar noon moves by \deltanoon{} seconds,
shifting from \evenoonhhmmss{} to \daynoonhhmmss{}
(allowing for rounding errors).
The sunrise moves by:

\formularise{}

that is, the sunrise moves by \deltarise{}
seconds, shifting from \everisehhmmss{} to \dayrisehhmmss{};
while the sunset moves by:

\formulaset{}

that is, moves by \deltaset{} seconds,
shifting from \evesethhmmss{} to \daysethhmmss{}.
In this case, the solar noon variation is much greater
that the half of the daylight variation, so the sunrise and
the sunset move in the same direction.

\vspace{5mm}

\subsection{Mea Culpa}

In the previous section, I made a logical mistake.
I have equated the solar noon with the precise instant located midway
between sunrise and sunset. This is nothing more than
an approximation. Let use go back to the March table.

\vspace{2mm}
\riseandset{20110320}{1}{3}{1}{20110321}
\vspace{2mm}

On 20th March, the daylight lasts 12 h 10 mn 09 s, that is, 43~809 seconds.
On 21st March, it lasts 12 h 14 mn 08 s, that is, 44~048 seconds
and on 22nd March, it lasts  12 h 18 mn 5 s, that is, 44~285 seconds.
This does not mean that on 20th March at 23 h 59, the daylight
duration is 43~809 seconds and that on 21st March
at 00 h 01 it becomes suddenly 44~048 seconds,
The daylight duration is a continuous function.
And since it varies by \deltalight{} seconds in
24 hours, it varies by 60 seconds in 6 hours.
This gives:

\vspace{2mm}
\begin{tabular}{|l|c|c|}
\hline
20~mars, noon & 43~809 & 12~h 10~mn 9~s \\
\hline
20~mars, 18~h & 43~869 & 12~h 11~mn 9~s \\
\hline
21~mars, 6~h & 43~988 & 12~h 13~mn 8~s \\
\hline
21~mars, noon & 44~048 & 12~h 14~mn 8~s \\
\hline
21~mars, 18~h & 44~108 & 12~h 15~mn 8~s \\
\hline
22~mars, 6~h & 44~225 & 12~h 17~mn 5~s \\
\hline
22~mars, noon & 44~285 & 12~h 18~mn 5~s \\
\hline
\end{tabular}
\vspace{2mm}

So, on 21st March, sunrise occurs 6 h 6 mn 34 s
before solar noon and sunset occurs 
6 h 7 mn 34 s seconds after solar noon.
From this, we infer that solar noon really occurs
at 12 h 7 mn 24 s instead of 12 h 7 mn 54 s
as printed in the table above.

The difference between solar noon and the
middle of the daylight is proportional both to the
daylight variation (maximum at the equinoxes)
and to the daylight duration (maximum at the summer
solstice). So the maximum difference occurs
around 19th April, when the daylight variation
is still approximately equal to its maximum value
and when the daylight have notably increased since
the equinox.

\vspace{2mm}
\riseandset{20110418}{1}{3}{1}{20110419}
\vspace{2mm}

On this date, at noon, the daylight lasts 50853 seconds
(that is, 14 h 7 mn 33 s) and it variation is \deltalight{} seconds.
So, at sunrise, the daylight is shorter by
\( \frac{228 \times 7}{24} = 66\) seconds and at sunset it is longer
by 66 seconds likewise. The morning lasts
\( \frac{50853 - 66}{2} = 25393 \) seconds and the afternoon lasts \( \frac{50853 + 66}{2} = 25459 \) seconds.
The end result is that solar noon occurs at 11:59:18
while the middle of daylight occurs at \daynoonhhmmss{}.

Does this logical and computation error invalidate
the reasoning from the previous section? No, because
as you can see, the result is an error no longer
than 33 seconds between solar noon and the
middle of the daylight. This is still much less
than the difference between solar noon and mean noon,
which can nearly reach 17 minutes.

%\vspace{2mm}
%\riseandset{20120101}{7}{52}{1}{20110419}
%\vspace{2mm}

In addition, the computation error is nearly zero
at the winter solstice, since the daylight variation
itself is nearly zero. Therefore, in the December table, 
the value printed for the solar noon is pretty much
the proper one.

\section{Solar Noon Variation}

%\vspace{2mm}
%\riseandset{20121216}{1}{15}{1}{20121225}
%\vspace{2mm}

The problem is now to understand why the solar noon
varies from one day to the next, in some cases with
a 30-second daily variation.

\subsection{Reference Case}

For the movements of Earth around itself and around
the sun, let us use the following simplificatons.

\begin{enumerate}

\item The year lasts exacty 360 days, instead of
365.25 and some.

\item Earth's orbit around the sun is a perfect circle.

\item The ecliptical plane of Earth's orbit is the same as
the equatorial plane. In other words, the axis of Earth's
spin is parallel to the axis of Earth revolution around the
sun.

\item We do not take into account the big scale variation, 
such as the variation of the angle between the equatorial plane
and the ecliptical plane, or the precession of equinoxes.

\item We do not take into account the influence of other celestial
bodies: the Moon, Jupiter, etc. With its mass
\directlua{affsci(\solarmass)} kg,
the sun applies a \directlua{force(\solarmass,\solardist)} N
to the Earth, while the Moon's attraction is 
only \directlua{force(\lunarmass,\lunardist)} N
and Jupiter applies a 
\directlua{force(\jupitermass,\jupiterdist)} N attraction.
So we can ignore the influence of other celestial bodies
and pay attention only to the Sun attraction.

\end{enumerate}

The first hypothesis and the last two ones are here only to 
simplify the computation and they will stay in effect for
the duration of the whole discussion. The second and third
hypothesis will be removed in turn to explain why the
solar noon does not coincidate with the mean noon.

If all five simplifications are in effect, Earth turns around
the sun in 360 days with a constant speed. So the orbital
movement is 1 degree per day. If we temporarily use a
set of coordinates centered on earth and using the
``fixed stars'', the sun moves one degree by day
with respect to the ``fixed stars''. To allow for this
1 degree per day movement and to have the sun at the
zenith at noon, Earth's rotation around its
axis is 361 degrees per 24 hours. Therefore,
on D+90 Earth has rotated for 90 turns and $\frac{1}{4}$,
on D+180 Earth has rotated for 180 turns and $\frac{1}{2}$
and on D+270 Earth has rotated for 270 turns and $\frac{3}{4}$.
See the picture below, which reverts to an heliocentric set of
coordinates.

\begin{mplibcode}
beginfig(1);
ua = 100;
z0 = (ua,0);
z90 = (0,ua);
z180 = (-ua,0);
z270 = (0,-ua);
draw z0{up} .. z90 .. z180 .. z270 .. z0;

rayon = 5;
tiret = 2;
for i = 0 step 90 until 270 :
  % cercle de la Terre
  draw (x[i] + rayon, y[i]) .. (x[i], y[i] + rayon) .. (x[i] - rayon, y[i]) .. (x[i], y[i] - rayon) .. cycle;
  % repère pour le Soleil au méridien
  draw (x[i], y[i]) + (rayon, 0) rotated (i + 180) --  (x[i], y[i]) + (rayon + tiret, 0) rotated (i + 180);
endfor;

% soleil
draw (rayon, 0) .. (0, rayon) .. (-rayon, 0) .. (0, - rayon) .. (rayon, 0);

ecart = rayon + 5;
drawarrow ( x90 - ecart,  y90) {down} .. {right} ( x90,  y90 - ecart);
drawarrow (x180 - ecart, y180) .. (x180, y180 - ecart) .. (x180 + ecart, y180);
drawarrow (x270 - ecart, y270) .. (x270, y270 - ecart) .. (x270 + ecart, y270) .. (x270, y270 + ecart);

label.rt ("D+0",    (x0 + rayon,           y0));
label.top("D+90",   (x90,                  y90 + rayon));
label.lft("D+180",  (x180 - rayon - ecart, y180));
label.bot("D+270",  (x270,                 y270 - rayon - ecart));
label.bot("Sun",    (0,                    0 - rayon));

endfig;
\end{mplibcode}

So, when Earth achieves a circular movement the plane of which
is the same as the equator's plane, the solar noon and the mean
noon occurs precisely at the same time.

\subsection{Kepler's Laws}

Kepler's first law tells us that Earth's orbit is an ellipse,
one focus of which is occupied by the solar system's 
center of inertia, that is, with a good approximation, the sun.

Kepler's second law tells us that the Earth's speed along
its sun-focused ellipse is such that the area-speed of the
sun-Earth vector is a constant.

Kepler's third law does not bring anything interesting in the
discussion.

Let us see what this would give with an ellipse with
a high eccentricity (0.5).

\begin{mplibcode}
beginfig(1);
ua = 100;
ub = 50 * sqrt(3);
z0 = (ua,0);
z61 = (0,ub);
z90 = (ua * cosd(116), ub * sind(116));
z180 = (-ua,0);
z270 = (x90, - y90);
z299 = (0,-ub);
draw z0{up} .. z61 .. z180 .. z299 .. z0;

rayon = 5;
tiret = 2;
for i = 0, 61, 90, 180, 270, 299 :
  % cercle de la Terre
  draw (x[i] + rayon, y[i]) .. (x[i], y[i] + rayon) .. (x[i] - rayon, y[i]) .. (x[i], y[i] - rayon) .. cycle;
endfor;

% soleil
xs = ua / 2;
ys = 0;
draw (xs + rayon, ys) .. (xs, ys + rayon) .. (xs - rayon, ys) .. (xs, ys - rayon) .. cycle;

label.rt  ("D+0",    (x0 + rayon,   y0));
label.urt ("D+61",   (x61,          y61 + rayon));
label.ulft("D+90",   (x90,          y90 + rayon));
label.lft ("D+180",  (x180 - rayon, y180));
label.llft("D+270",  (x270,         y270 - rayon));
label.lrt ("D+299",  (x299,         y299 - rayon));
label.lrt ("Sun",    (xs,           ys - rayon));

for i = 0 step 90 until 270 :
  draw (xs, ys) -- (x[i], y[i]);
endfor

endfig;
\end{mplibcode}

Since the sun is placed in one of the ellipse's foci, on the right of
the drawing, we have to shift the D+90 and D+270 points to the
left, so that the areas (Sun, D+0, D+90), (Sun, D+90, D+180),
(Sun, D+180, D+270) and (Sun, D+270, D+0) are equivalent.
So Earth reaches the end of the ellipse's small axis on D+61 instead
of D+90, and the other end on D+299 instead of D+270.

\begin{mplibcode}
beginfig(1);
ua = 100;
ub = 50 * sqrt(3);
z0 = (ua,0);
z61 = (0,ub);
z90 = (ua * cosd(116), ub * sind(116));
z180 = (-ua,0);
z270 = (x90, - y90);
z299 = (0,-ub);
draw z0{up} .. z61 .. z180 .. z299 .. z0;

rayon = 5;
tiret = 2;
for i = 0, 61, 90, 180, 270, 299 :
  % cercle de la Terre
  draw (x[i] + rayon, y[i]) .. (x[i], y[i] + rayon) .. (x[i] - rayon, y[i]) .. (x[i], y[i] - rayon) .. cycle;
  % repère pour le Soleil au méridien
  draw (x[i], y[i]) + (rayon, 0) rotated (i + 180) --  (x[i], y[i]) + (rayon + tiret, 0) rotated (i + 180);
endfor;

% soleil
xs = ua / 2;
ys = 0;
draw (xs + rayon, ys) .. (xs, ys + rayon) .. (xs - rayon, ys) .. (xs, ys - rayon) .. cycle;

ecart = rayon + 5;
drawarrow ( x61 - ecart,  y61) {down} ..  ( (x61, y61) + (ecart, 0) rotated 241);
drawarrow ( x90 - ecart,  y90) {down} .. {right} ( x90,  y90 - ecart);
drawarrow (x180 - ecart, y180) .. (x180, y180 - ecart) .. (x180 + ecart, y180);
drawarrow (x270 - ecart, y270) .. (x270, y270 - ecart) .. (x270 + ecart, y270) .. (x270, y270 + ecart);
drawarrow (x299 - ecart, y299) .. (x299, y299 - ecart) .. (x299 + ecart, y299) .. (x299, y299 + ecart) .. ( (x299, y299) + (ecart, 0) rotated 109);

label.rt  ("D+0",    (x0 + rayon,           y0));
label.urt ("D+61",   (x61,                  y61 + rayon));
label.ulft("D+90",   (x90,                  y90 + rayon));
label.lft ("D+180",  (x180 - rayon - ecart, y180));
label.llft("D+270",  (x270,                 y270 - rayon - ecart));
label.lrt ("D+299",  (x299,                 y299 - rayon - ecart));
label.bot ("Sun",    (xs,                   ys - rayon));

endfig;
\end{mplibcode}

On day D+61 at mean noon, Earth has rotated 61 turns plus 61 degrees, 
but it does not aim at the sun. It must rotate an additional 59 degrees
to aim at the sun and reach solar noon. Likewise, on day D+90 at mean
noon, Earth has rotated 90 turns plus 90 degrees and has to rotate
an additional 46 degrees to reach solar noon.

On the other side (literally), on day D+270, the solar noon occurs before
the mean noon: the solar noon occurs after 270 turns plus 224 degrees 
and the mean noon occurs after 270 turns plus 270 degrees, that is,
46 degrees after the solar noon.

\begin{mplibcode}
beginfig(1);

rayon = 20;
rayonf = 25;
pair meannoon, solarnoon;

z61 = (0, 0);
draw (z61 + (rayon, 0)) .. (z61 + (0, rayon)) .. (z61 + (-rayon, 0)) .. (z61 + (0, -rayon)) .. cycle;
label.top("D+61", z61 + (0, rayon));

meannoon   := z61 + (rayonf, 0) rotated 241;
solarnoon := z61 + (rayonf, 0) rotated 300;
draw (z61 + (rayon, 0) rotated 241) -- meannoon;
draw (z61 + (rayon, 0) rotated 300) -- solarnoon;
label.llft("Mean noon",  meannoon);
label.lrt ("Solar noon", solarnoon);
drawarrow (z61 - (rayonf, 0)) {down} .. meannoon;
drawarrow meannoon { dir 331 } .. { dir 30} solarnoon;

z90 = (100, 0);
draw (z90 + (rayon, 0)) .. (z90 + (0, rayon)) .. (z90 + (-rayon, 0)) .. (z90 + (0, -rayon)) .. cycle;
label.top("D+90", z90 + (0, rayon));
meannoon   := z90 + (rayonf, 0) rotated 270;
solarnoon := z90 + (rayonf, 0) rotated 314;
draw (z90 + (rayon, 0) rotated 314) -- solarnoon;
draw (z90 + (rayon, 0) rotated 270) -- meannoon;
label.bot ("Mean noon",  meannoon);
label.lrt ("Solar noon", solarnoon);
drawarrow (z90 - (rayonf, 0)) {down} .. meannoon;
drawarrow meannoon { dir 0 } .. { dir 44 }  solarnoon;

z270 = (200, 0);
draw (z270 + (rayon, 0)) .. (z270 + (0, rayon)) .. (z270 + (-rayon, 0)) .. (z270 + (0, -rayon)) .. cycle;
label.top("D+270", z270 + (0, rayonf + 10));
solarnoon := z270 + (rayonf, 0) rotated 44;
meannoon   := z270 + (rayonf, 0) rotated 90;
draw (z270 + (rayon, 0) rotated 44) -- solarnoon;
draw (z270 + (rayon, 0) rotated 90) -- meannoon;
label.lft("Mean noon",  meannoon);
label.rt ("Solar noon", solarnoon);
drawarrow (z270 - (rayonf, 0)) {down} .. (z270 - (0, rayonf)) .. (z270 + (rayonf, 0)) .. solarnoon;
drawarrow solarnoon { dir 134 } .. { dir 180}  meannoon;

z299 = (350, 0);
draw (z299 + (rayon, 0)) .. (z299 + (0, rayon)) .. (z299 + (-rayon, 0)) .. (z299 + (0, -rayon)) .. cycle;
label.top("D+299", z299 + (0, rayonf + 10));

solarnoon := z299 + (rayonf, 0) rotated 63;
meannoon   := z299 + (rayonf, 0) rotated 119;
draw (z299 + (rayon, 0) rotated 63) -- solarnoon;
draw (z299 + (rayon, 0) rotated 119) -- meannoon;
label.urt ("Solar noon", solarnoon);
label.ulft("Mean noon",  meannoon);
%drawarrow (z299 - (rayonf, 0)) {down} .. (z299 - (0, rayonf)) .. (z299 + (rayonf, 0)) .. solarnoon;
drawarrow (z299 - (rayonf, 0)) {down} .. (z299 - (0, rayonf)) .. solarnoon;
drawarrow solarnoon { dir 153 } .. { dir 209 } meannoon;

endfig;
\end{mplibcode}

In these examples, the 46- and 59-degree angles are equivalent
to a 3- and 4-hour delay respectively. This is very much greater
than the delays that happen on the real world Earth. But at the 
same time, the real world orbit's eccentricity is only 0.02
while the description above posits a 0.5 eccentricity.

\subsection{Influence of the Orbit Slope}

The second simplification is reinstated, while the third simplification
is removed. The discussion is a bit harder to follow, because it
requires to think in 3-D, which is not convenient on
a 2-D sheet of paper.

Let us start from the five simplifications are in effect. Then,
we rotate the orbital plane \textit{while keeping
the equatorial plane parallel to itself}, then we project
the new orbit orthogonally on the former orbit's plane, 
\textit{while still keeping the equatorial plane parallel 
to itself}.

\begin{mplibcode}
beginfig(1);

ua = 100;
rayon = 5;
tiret = 2;

% Terre vue de profil
def terreprofil(expr z) =
fill z + (rayon, 0) .. z + (0, rayon) .. z - (rayon, 0) .. z - (0, rayon) .. cycle;
draw z - (rayon + tiret, 0) -- z + (rayon + tiret, 0);
enddef;

% transformation vue de profil
xprof = -200;
draw (xprof, ua) -- (xprof, -ua);
draw (xprof, 0) + (0, ua) rotated 60 -- (xprof, 0) + (0, -ua) rotated 60;
terreprofil( (xprof, ua ) );
terreprofil( (xprof, 0 ) + (0, ua) rotated 60 );
terreprofil( (xprof, ua / 2 ) );
terreprofil( (xprof, - ua ) );
terreprofil( (xprof, 0 ) - (0, ua) rotated 60 );
terreprofil( (xprof, - ua / 2 ) );

drawarrow  (xprof - rayon - 3 * tiret, ua) {left} .. {dir 240} ((xprof, 0 ) + (0 + rayon + 3 * tiret, ua + tiret) rotated 60);
drawarrow  (xprof + rayon + 3 * tiret, 0 ) + (0, ua) rotated 60 --  (xprof - rayon - 3 * tiret, ua / 2);
drawarrow  (xprof + rayon + 3 * tiret, - ua) {right} .. {dir 60} ((xprof, 0 ) - (0 + rayon + 3 * tiret, ua + tiret) rotated 60);
drawarrow  (xprof - rayon - 3 * tiret, 0 ) - (0, ua) rotated 60 --  (xprof + rayon + 3 * tiret, - ua / 2);

% transformation vue de face
z0 = (ua,0);
z90 = (0,ua);
z180 = (-ua,0);
z270 = (0,-ua);

pair zf[];
path orbite;
zf90 = (0, ua / 2);
zf270 = (0, - ua / 2);
orbite = z0{up} .. z90 .. z180 .. z270 .. z0;
draw orbite;
draw orbite yscaled 0.5;

for i = 0 step 90 until 270 :
  % cercle de la Terre
  draw (x[i] + rayon, y[i]) .. (x[i], y[i] + rayon) .. (x[i] - rayon, y[i]) .. (x[i], y[i] - rayon) .. cycle;
  % repère pour le Soleil au méridien
  draw (x[i], y[i]) + (rayon, 0) rotated (i + 180) --  (x[i], y[i]) + (rayon + tiret, 0) rotated (i + 180);
endfor;

% soleil
draw (rayon, 0) .. (0, rayon) .. (-rayon, 0) .. (0, - rayon) .. (rayon, 0);

% Nouveau J+90
draw ((rayon, 0) .. (0, rayon) .. (- rayon, 0) .. (0, - rayon) .. cycle) shifted zf90;
draw ((0, - rayon) -- (0, - rayon - tiret)) shifted (x90, y90 / 2);

drawarrow (z90 - (0, 2 * rayon)) -- (zf90 + (0, 2 * rayon));

% Nouveau J+270
draw ((rayon, 0) .. (0, rayon) .. (- rayon, 0) .. (0, - rayon) .. cycle) shifted zf270;
draw ((0, + rayon) -- (0, + rayon + tiret)) shifted zf270;

drawarrow (z270 + (0, 2 * rayon)) -- (zf270 - (0, 2 * rayon));

ecart := rayon + 5;
drawarrow ( x90 - ecart,  y90) {down} .. {right} ( x90,  y90 - ecart);
drawarrow (x180 - ecart, y180) .. (x180, y180 - ecart) .. (x180 + ecart, y180);
drawarrow (x270 - ecart, y270) .. (x270, y270 - ecart) .. (x270 + ecart, y270) .. (x270, y270 + ecart);

label.rt ("D+0",    (x0 + rayon,           y0));
label.top("D+90",   (x90,                  y90 + rayon));
label.lft("D+180",  (x180 - rayon - ecart, y180));
label.bot("D+270",  (x270,                 y270 - rayon - ecart));
label.bot("Sun",    (0,                    0 - rayon));

endfig;
\end{mplibcode}

The end result is that the orbit's circle is contracted along
the vertical axis, which gives an ellipse. All the while,
Earth's rotation around its pole axis stays the same, 
since the pole-to-pole axis was constantly kept parallel to
itself.

Since we use an exaggerated slope of 60 degrees, the circle is
contracted by half. With the real world 23-degree slope,
the contraction would be only 8\%.

Let us now consider the days D+45, D+135, D+225 and D+315.

\begin{mplibcode}
beginfig(1);

ua = 100;
rayon = 5;
tiret = 2;
% transformation vue de face
z0   = ( ua,  0);
z90  = (  0, ua);
z180 = (-ua,  0);
z270 = (  0,-ua);

path orbite;
orbite = z0{up} .. z90 .. z180 .. z270 .. z0;

z45  = point 0.5 of orbite;
z135 = point 1.5 of orbite;
z225 = point 2.5 of orbite;
z315 = point 3.5 of orbite;

pair zf[];
zf90 = (0, ua / 2);
xf90 = 0;
yf90 = ua / 2;
zf270 = (0, - ua / 2);
xf270 = 0;
yf270 = - ua / 2;
zf45  = z45  yscaled 0.5;
zf135 = z135 yscaled 0.5;
zf225 = z225 yscaled 0.5;
zf315 = z315 yscaled 0.5;

draw orbite;
draw orbite yscaled 0.5;

for i = 0 step 45 until 315 :
  % cercle de la Terre
  draw (x[i] + rayon, y[i]) .. (x[i], y[i] + rayon) .. (x[i] - rayon, y[i]) .. (x[i], y[i] - rayon) .. cycle;
  % repère pour le Soleil au méridien
  draw (x[i], y[i]) + (rayon, 0) rotated (i + 180) --  (x[i], y[i]) + (rayon + tiret, 0) rotated (i + 180);
endfor;

% soleil
draw (rayon, 0) .. (0, rayon) .. (-rayon, 0) .. (0, - rayon) .. (rayon, 0);

% Nouveau J+45
draw ((rayon, 0) .. (0, rayon) .. (- rayon, 0) .. (0, - rayon) .. cycle) shifted zf45;
draw ((- rayon, 0) -- (- rayon - tiret, 0)) rotated 45 shifted (x45, y45 / 2);

drawarrow (z45 - (0, 2 * rayon)) -- (zf45 + (0, 2 * rayon));

% Nouveau J+90
draw ((rayon, 0) .. (0, rayon) .. (- rayon, 0) .. (0, - rayon) .. cycle) shifted zf90;
draw ((0, - rayon) -- (0, - rayon - tiret)) shifted (x90, y90 / 2);

drawarrow (z90 - (0, 2 * rayon)) -- (zf90 + (0, 2 * rayon));

% Nouveau J+135
draw ((rayon, 0) .. (0, rayon) .. (- rayon, 0) .. (0, - rayon) .. cycle) shifted zf135;
draw ((- rayon, 0) -- (- rayon - tiret, 0)) rotated 135 shifted (x135, y135 / 2);

drawarrow (z135 - (0, 2 * rayon)) -- (zf135 + (0, 2 * rayon));

% Nouveau J+225
draw ((rayon, 0) .. (0, rayon) .. (- rayon, 0) .. (0, - rayon) .. cycle) shifted zf225;
draw ((- rayon, 0) -- (- rayon - tiret, 0)) rotated 225 shifted (x225, y225 / 2);

drawarrow (z225 + (0, 2 * rayon)) -- (zf225 - (0, 2 * rayon));

% Nouveau J+270
draw ((rayon, 0) .. (0, rayon) .. (- rayon, 0) .. (0, - rayon) .. cycle) shifted zf270;
draw ((0, + rayon) -- (0, + rayon + tiret)) shifted zf270;

drawarrow (z270 + (0, 2 * rayon)) -- (zf270 - (0, 2 * rayon));

% Nouveau J+315
draw ((rayon, 0) .. (0, rayon) .. (- rayon, 0) .. (0, - rayon) .. cycle) shifted zf315;
draw ((- rayon, 0) -- (- rayon - tiret, 0)) rotated 315 shifted (x315, y315 / 2);

drawarrow (z315 + (0, 2 * rayon)) -- (zf315 - (0, 2 * rayon));

ecart := rayon + 5;
%drawarrow ( x90  - ecart,  y90)  {down} .. {right} ( x90,   y90  - ecart);
%drawarrow ( xf90 - ecart,  yf90) {down} .. {right} ( xf90,  yf90 - ecart);
%drawarrow (x180 - ecart, y180) .. (x180, y180 - ecart) .. (x180 + ecart, y180);
%drawarrow (x270  - ecart, y270 ) .. (x270 , y270  - ecart) .. (x270  + ecart, y270 ) .. (x270 , y270  + ecart);
%drawarrow (xf270 - ecart, yf270) .. (xf270, yf270 - ecart) .. (xf270 + ecart, yf270) .. (xf270, yf270 + ecart);

label.rt ("D+0",    (x0 + rayon,           y0));
label.rt ("D+45",   (x45 + rayon,          y45));
label.top("D+90",   (x90,                  y90 + rayon));
label.lft("D+135",  (x135 - rayon - ecart, y135));
label.lft("D+180",  (x180 - rayon - ecart, y180));
label.lft("D+225",  (x225 - rayon - ecart, y225));
label.bot("D+270",  (x270,                 y270 - rayon - ecart));
label.rt ("D+315",  (x315 + rayon,         y315));
label.bot("Sun",    (0,                    0 - rayon));

endfig;
\end{mplibcode}

With this transformation, we notice that mean noon and solar
noon occurs at the same time for D+0, D+90, D+180 and D+270. But
this is not the case with the other days, especially 
D+45, D+135, D+225 and D+315.


\begin{mplibcode}
beginfig(1);

rayon = 20;
rayonf = 25;
pair meannoon, solarnoon;

z45 = (0, 0);
draw (z45 + (rayon, 0)) .. (z45 + (0, rayon)) .. (z45 + (-rayon, 0)) .. (z45 + (0, -rayon)) .. cycle;
label.top("D+45", z45 + (0, rayon));
meannoon   := z45 + (rayonf, 0) rotated 225;
solarnoon := z45 + (rayonf, 0) rotated 207;
draw (z45 + (rayon, 0) rotated 225) -- meannoon;
draw (z45 + (rayon, 0) rotated 207) -- solarnoon;
label.bot ("Mean noon",  meannoon);
label.lft ("Solar noon", solarnoon);
drawarrow (z45 - (rayonf, 0)) {down} .. solarnoon;
drawarrow solarnoon { dir 297 } .. { dir 315} meannoon;

z135 = (70, 0);
draw (z135 + (rayon, 0)) .. (z135 + (0, rayon)) .. (z135 + (-rayon, 0)) .. (z135 + (0, -rayon)) .. cycle;
label.top("D+135", z135 + (0, rayon));
meannoon   := z135 + (rayonf, 0) rotated 315;
solarnoon := z135 + (rayonf, 0) rotated 333;
draw (z135 + (rayon, 0) rotated 315) -- meannoon;
draw (z135 + (rayon, 0) rotated 333) -- solarnoon;
label.lrt("Mean noon",  meannoon);
label.rt ("Solar noon", solarnoon);
drawarrow (z135 - (rayonf, 0)) {down} .. meannoon;
drawarrow meannoon { dir 45 } .. { dir 63}  solarnoon;

z225 = (180, 0);
draw (z225 + (rayon, 0)) .. (z225 + (0, rayon)) .. (z225 + (-rayon, 0)) .. (z225 + (0, -rayon)) .. cycle;
label.top("D+225", z225 + (0, rayon));
solarnoon := z225 + (rayonf, 0) rotated  27;
meannoon   := z225 + (rayonf, 0) rotated  45;
draw (z225 + (rayon, 0) rotated  27) -- solarnoon;
draw (z225 + (rayon, 0) rotated  45) -- meannoon;
label.rt  ("Solar noon", solarnoon);
label.urt ("Mean noon",  meannoon);
drawarrow (z225 - (rayonf, 0)) {down} .. solarnoon;
drawarrow solarnoon { dir 217 } .. { dir 135} meannoon;

z315 = (350, 0);
draw (z315 + (rayon, 0)) .. (z315 + (0, rayon)) .. (z315 + (-rayon, 0)) .. (z315 + (0, -rayon)) .. cycle;
label.top("D+315", z315 + (0, rayonf));
meannoon   := z315 + (rayonf, 0) rotated 135;
solarnoon := z315 + (rayonf, 0) rotated 153;
draw (z315 + (rayon, 0) rotated 135) -- meannoon;
draw (z315 + (rayon, 0) rotated 153) -- solarnoon;
label.ulft("Mean noon",  meannoon);
label.lft ("Solar noon", solarnoon);
drawarrow (z315 - (rayonf, 0)) {down} .. (z315 - (0, rayonf)) .. (z315 + (rayonf, 0)) .. meannoon;
drawarrow meannoon { dir 225 } .. { dir 243}  solarnoon;

endfig;
\end{mplibcode}

We see that the solar noon occurs before the mean noon from D+0 to D+90
and then again from D+180 to D+270, while the mean noon occurs before
the solar noon from D+90 to D+180 and again from D+270 to D+360.
Instead of a periodical variation with a 1-year period, we have now
a variation with a 1-half-year period.

\subsection{In Summary}

With the values used in the discussion, 0.5 eccentricity and 
60-degree slope, merging the two phenomenons is difficult.
But with the real world values, eccentricity less than 0.02
and 23-degree slope, we can assimilate the variations to
sine waves with a rather small amplitude
and add them. Attention, the D+0 days are not the same for
both phenomenons. For the eccentricity, D+0 is 3rd of January,
when Earth is at its perihelion. For the angle between the equatorial
and orbital plane, D+0 is either equinox, around 21st
March and 21st September.

The real variation curve is given in page 32 of
``Histoire de l'Heure en France'', 
by Jacques Gapaillard, published by Vuibert -- ADAPT. 
The equation of time (gap between the solar noon and the
mean noon) spans from a delay of 14 mn and 15 s of the 
solar noon with respect to the mean noon, to a 16 mn and
10 s advance of the solar noon with respect to the mean noon.

\section{Annex}

This text is published with the Creative Commons License
Attribution - No Commercial Use - Share-Alike 
3.0 : CC BY-NC-SA.
The programs embedded in this document are published
under double license GPL + Artistic.
Copyright (c) 2012-2013, Jean Forget.

\end{texte}
\end{document}
